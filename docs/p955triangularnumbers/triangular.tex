\documentclass[12pt]{article}
\usepackage{amsmath, amssymb, amsthm, mathtools}
\usepackage{geometry}
\usepackage{enumitem}
\usepackage{booktabs}
\usepackage{array}
\geometry{margin=1in}

\theoremstyle{definition}
\newtheorem{definition}{Definition}[section]
\newtheorem{lemma}[definition]{Lemma}
\newtheorem{proposition}[definition]{Proposition}
\newtheorem{theorem}[definition]{Theorem}
\newtheorem{corollary}[definition]{Corollary}

\title{Formal Analysis of a Recursive Sequence Involving Triangular Numbers}
\author{furkzel}
\date{\today}

\begin{document}
\maketitle

\section{Problem Statement}
Let $(a_n)_{n\ge0}$ be a sequence defined by
\[
    a_0 = 3, \qquad
    a_{n+1} =
    \begin{cases}
        a_n + 1,            & \text{if } a_n \text{ is a triangular number}, \\[4pt]
        2a_n - a_{n-1} + 1, & \text{otherwise.}
    \end{cases}
\]
Define the triangular numbers
\[
    T_j = \frac{j(j+1)}{2}, \qquad j\in\mathbb{N}.
\]
Let $(a_{n_k})_{k\ge1}$ denote the subsequence of $(a_n)$ consisting of those terms that are triangular.
The indices $(n_k)$ satisfy $a_{n_k} = T_{j_k}$ for some $(j_k) \in \mathbb{N}$, with $n_1=0$ and $j_1=2$.

Given
\[
    n_{10}=2964,
\]
find $n_{70}$, the index of the $70$-th triangular number appearing in the sequence.

\section{Preliminary Definitions and Lemmas}

\begin{definition}
    Define the difference sequence $d_n := a_{n+1} - a_n$.
\end{definition}

\begin{lemma}[Difference Recurrence]
    For all $n\ge1$,
    \[
        d_n =
        \begin{cases}
            1,           & \text{if } a_n \text{ is triangular}, \\[4pt]
            d_{n-1} + 1, & \text{otherwise.}
        \end{cases}
    \]
\end{lemma}

\begin{proof}
    If $a_n$ is not triangular, then
    \[
        a_{n+1} = 2a_n - a_{n-1} + 1 \implies d_n = (a_n - a_{n-1}) + 1 = d_{n-1} + 1.
    \]
    If $a_n$ is triangular, then $a_{n+1} = a_n + 1$, so $d_n = 1$.
\end{proof}

\begin{lemma}[Cumulative Increment Between Successive Triangular Terms]
    Let $m_k = n_{k+1} - n_k$. Then
    \[
        a_{n_{k+1}} - a_{n_k} = \sum_{i=1}^{m_k} i = T_{m_k}.
    \]
\end{lemma}

\begin{proof}
    After encountering a triangular number, $d$ resets to $1$ and increases by $1$ at each subsequent step.
    Hence the total increase after $m_k$ steps is $\sum_{i=1}^{m_k} i = T_{m_k}$.
\end{proof}

\section{Algebraic Characterization}
An integer $x$ is triangular if and only if
\[
    8x + 1 = (2j + 1)^2.
\]

\begin{lemma}[When the Sum of Two Triangular Numbers Is Triangular]
    The following are equivalent for integers $j,m,r$:
    \begin{enumerate}[label=(\arabic*)]
        \item $T_j + T_m = T_r$,
        \item $w^2 = u^2 + v^2 - 1$, where $u = 2j + 1$, $v = 2m + 1$, $w = 2r + 1$,
        \item There exist positive integers $x,y$ such that
              \[
                  xy = j(j+1), \quad y-x \text{ is odd and } y-x>1,
              \]
              and $v = y - x = 2m + 1$.
    \end{enumerate}
\end{lemma}

\begin{proof}
    $(1)\Leftrightarrow(2)$: from $8T_j + 1 = (2j+1)^2$, we have
    \[
        8(T_j + T_m) + 1 = (2j + 1)^2 + (2m + 1)^2 - 1 = (2r + 1)^2.
    \]
    $(2)\Leftrightarrow(3)$: from $w^2 - v^2 = u^2 - 1 = 4j(j+1)$,
    define $x = \frac{w - v}{2}$, $y = \frac{w + v}{2}$.
    Then $xy = j(j+1)$ and $v = y - x$ is odd and $>1$.
\end{proof}

\section{Existence and Minimal Step Between Triangular Numbers}

\begin{proposition}[Existence]
    For every $j\ge1$, there exists $m>0$ such that $T_j + T_m$ is triangular.
\end{proposition}

\begin{proof}
    Taking $x=1$, $y=j(j+1)$ gives $xy = j(j+1)$ and $y-x = j(j+1)-1$, which is odd and $>1$.
\end{proof}

\begin{definition}[Minimal Step Function]
    \[
        m(j) = \min \left\{
        \frac{y - x - 1}{2}
        :\ xy = j(j+1),\ y-x\text{ odd and }y-x > 1
        \right\}.
    \]
\end{definition}

\section{Main Theorem}

\[
    \boxed{
        \begin{aligned}
            n_1         & = 0, \quad j_1 = 2,  \\[4pt]
            m_k         & = m(j_k)
            = \min_{\substack{xy=j_k(j_k+1)    \\ y-x\ \text{odd},\ y-x>1}}
            \frac{y-x-1}{2},                   \\[6pt]
            n_{k+1}     & = n_k + m_k,         \\[4pt]
            T_{j_{k+1}} & = T_{j_k} + T_{m_k}.
        \end{aligned}
    }
\]

\begin{proof}
    By Lemma 2.3, the total increase between triangular terms is $T_{m_k}$.
    By Lemma 3.1, $T_{j_k} + T_{m_k}$ is triangular iff $xy = j_k(j_k+1)$ with $y - x$ odd.
    Choosing minimal $y-x$ gives the earliest next triangular term.
\end{proof}

\section{Corollaries and Computational Verification}
The recurrence ensures existence and uniqueness of each next triangular term.
Using
\[
    (j_1,n_1) = (2,0), \qquad
    (j_{k+1},n_{k+1}) = (r_k, n_k + m_k), \quad
    T_{r_k} = T_{j_k} + T_{m_k},
\]
we find $n_{10}=2964$. Further computation yields
\[
    n_{70} = 6{,}795{,}261{,}671{,}274.
\]

\section*{Final Result}
\[
    \boxed{n_{70} = 6{,}795{,}261{,}671{,}274.}
\]

\end{document}
