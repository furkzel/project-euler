\documentclass[12pt]{article}
\usepackage{amsmath, amssymb, amsthm, mathtools}
\usepackage{geometry}
\usepackage{enumitem}
\usepackage{booktabs}
\usepackage{array}
\geometry{margin=1in}

\theoremstyle{definition}
\newtheorem{definition}{Definition}[section]
\newtheorem{lemma}[definition]{Lemma}
\newtheorem{proposition}[definition]{Proposition}
\newtheorem{theorem}[definition]{Theorem}
\newtheorem{corollary}[definition]{Corollary}

\title{Rolling Ellipse Path Length: Support Function and Convex Analysis Approach}
\author{furkzel}
\date{\today}

\begin{document}
\maketitle

\section{Problem Statement}
An ellipse $E(a, b)$ is given at its initial position by equation:

\[\frac{(x - a)^2}{a^2} + \frac{(y - b)^2}{b^2} = 1\]

The ellipse rolls without slipping along the $x$ axis for one complete turn. Interestingly, the length of the curve generated by a focus is independent from the size of the minor axis:<br> $F(a,b) = 2 \pi \max(a,b)$.

This is not true for the curve generated by the ellipse center. Let $C(a, b)$ be the length of the curve generated by the center of the ellipse as it rolls without slipping for one turn.

You are given $C(2, 4) \approx 21.38816906$

Find $C(1, 4) + C(3, 4)$. Give your answer rounded to $8$ digits behind the decimal point in the form $ab.cdefghij$.

\section{Definitions and Preliminaries}

We're giving some definitions and concepts for solving this problem.

\begin{itemize}
    \item \textbf{Convex Curve:} A curve $C \subset \mathbb{R}^2$ is strictly convex if for any two points on the curve, the line segment connecting them lies entirely outside the curve, except at the endpoints.
    \item \textbf{Support Function:} For a strictly convex, closed curve $C$ of class $C^2$, the support function $h: \mathbb{R} \to (0, \infty)$ is defined as:
          \[
              h(\theta) := \max_{x \in C} \langle x, n(\theta) \rangle \quad \text{where} \quad n(\theta) := (\cos \theta, \sin \theta).
          \]
          This represents the distance from the origin to the tangent line in the direction $n(\theta)$.
    \item \textbf{Radius of Curvature:} The radius of curvature $\rho(\theta)$ at a point on the curve is given by the Cauchy formula:
          \[
              \rho(\theta) = h(\theta) + h''(\theta)
          \]
          where $h''(\theta)$ is the second derivative of the support function with respect to $\theta$.
    \item $C \subset \mathbb{R}^2:C^2$, is a strictly convex, closed curve of class $ C^2 $, with its center at the origin, parameterized by the tangent direction angle $ \theta \in \mathbb{R} $.
\end{itemize}

\section{Lemmas and Theorems}
\begin{lemma}[Cauchy Formula]
    For a strictly convex, closed curve $C$ of class $C^2$, the radius of curvature $\rho(\theta)$ at a point on the curve is given by:
    \[
        \rho(\theta) = h(\theta) + h''(\theta)
    \]
    where $h(\theta)$ is the support function and $h''(\theta)$ is its second derivative with respect to $\theta$.
\end{lemma}

\begin{proof}
    Classical support parametrization of convex curve.
    \[
        \gamma(\theta) = h(\theta)n(\theta) + h'(\theta)n^{\perp}(\theta, \qquad t(\theta)=(-\sin\theta, \cos\theta))
    \]
    Differentiating $\gamma(\theta)$ with respect to $\theta$ gives:
    \[
        \gamma'(\theta) = (h(\theta) + h''(\theta)) t(\theta)
    \]
    The magnitude of $\gamma'(\theta)$ is:
    \[
        \|\gamma'(\theta)\| = h(\theta) + h''(\theta)
    \]
    Thus, the radius of curvature $\rho(\theta)$ is given by:
    \[
        \rho(\theta) = h(\theta) + h''(\theta)
    \]
\end{proof}

\begin{lemma}[Rolling Kinematics]
    When a strictly convex curve rolls without slipping along a straight line, the horizontal displacement differential taken on the ground is equal to the arc length differential of the curve's circumference.
\end{lemma}

\begin{proof}
    Let $C$ be a strictly convex curve rolling without slipping along the $x$-axis. At any point of contact, the tangent line to the curve is horizontal. The arc length differential $ds$ of the curve corresponds to the horizontal displacement differential $dx$ on the ground.

    Since there is no slipping, the distance traveled by the point of contact on the ground must equal the arc length traversed by the curve. Therefore, we have:
    \[
        dx = ds
    \]
    This establishes that the horizontal displacement differential taken on the ground is equal to the arc length differential of the curve's circumference.
\end{proof}

\begin{theorem}[Center Rolling Path Length]
    The length of the curve generated by the center of a strictly convex ellipse rolling without slipping along the $x$-axis for one complete turn is given by:
    \[
        C(a,b)=\int_{0}^{2\pi}\sqrt{\rho(\theta)^2+\left(\frac{dy}{d\theta}\right)^2}d\theta
    \]
    where $\rho(\theta)$ is the radius of curvature and $y(\theta)$ is the height of the center above the ground.
\end{theorem}

\begin{proof}
    From the rolling kinematics lemma, we know that the horizontal displacement differential $dx$ on the ground is equal to the arc length differential of the ellipse's circumference. Using the support function $h(\theta)$ for the ellipse, we can express the radius of curvature $\rho(\theta)$ as:
    \[
        \rho(\theta) = h(\theta) + h''(\theta)
    \]
    The height of the center above the ground when the tangent line is horizontal is given by:
    \[
        y(\theta) = h(\theta)
    \]
    Therefore, the derivative of $y(\theta)$ with respect to $\theta$ is:
    \[
        \frac{dy}{d\theta} = h'(\theta)
    \]
    The length of the curve generated by the center of the ellipse can be expressed as:
    \[
        C(a,b)=\int_{0}^{2\pi}\sqrt{\left(\frac{dx}{d\theta}\right)^2+\left(\frac{dy}{d\theta}\right)^2}d\theta
    \]
    Substituting $\frac{dx}{d\theta}=\rho(\theta)$ and $\frac{dy}{d\theta}=h'(\theta)$, we get:
    \[
        C(a,b)=\int_{0}^{2\pi}\sqrt{\rho(\theta)^2+\left(h'(\theta)\right)^2}d\theta
    \]
    This completes the proof.
\end{proof}

\section{Solution Sketch and Key Approaches}
To solve the problem of finding the length of the curve generated by the center of a rolling ellipse, we can utilize concepts from convex analysis and support functions, I guess.

There is an ellipse and It rolls without slipping along the x-axis. We can define a support function for the ellipse, which describes the distance from the origin to the tangent line in a given direction, as below:

\[
    h(\theta) = \sqrt{a^2\cos^2\theta+b^2\sin^2\theta}
\]

The rolling kinematics are standard:

When the tangent line is horizontal (i.e., the ground line), the center height of the ellipse is directly
\[
    y(\theta) = h(\theta)
\]

\textit{(the distance between the support line and the center).}

In lossless rolling, the horizontal displacement differential taken on the ground is equal to the arc length differential of the ellipse's circumference. With the outer normal parameter, this is

\[
    \frac{dx}{d\theta}=\rho(\theta), \qquad \rho(\theta)=h(\theta)+h''(\theta)
\]

\textit{(Cauchy formula: $\rho$ is curvate radius)}

Combining these, we can express the center path length $C(a,b)$ as an integral over one full rotation:

\[
    C(a,b)=\int_{0}^{2\pi}\sqrt{\left(\frac{dx}{d\theta}\right)^2+\left(\frac{dy}{d\theta}\right)^2}d\theta=\int_{0}^{2\pi}\sqrt{\rho(\theta)^2+\left(\frac{dy}{d\theta}\right)^2}d\theta
\]

And same result, we can get using parametric equations of the ellipse and numerical integration. At the touch point, $P(t) = (a\cos t, b\sin t)$:
\begin{itemize}
    \item Velocity of the touch point: $\frac{dX}{dt}= \sqrt{a^2\sin^2t + b^2 \cos^2t}$.
    \item Height of the center, the distance between the center and the tangent line $x\frac{\cos t}{a}+y\frac{\sin t}{b}=1$:
          \[
              Y(t) = \frac{ab}{\sqrt{a^2\sin^2t + b^2 \cos^2t}}
          \]
          Therefore, its derivative is:
          \[
              \frac{dY}{dt} = -\frac{ab(a^2-b^2)\sin t \cos t}{(a^2\sin^2t + b^2 \cos^2t)^{3/2}}
          \]
\end{itemize}

With this parametric approach, we can express the center path length as:

\[
    C(a,b)=\int_{0}^{2\pi}\sqrt{\big(a^2\sin^2 t+b^2\cos^2 t\big)
        +\left(\frac{dY}{dt}\right)^2}dt ;
\]

which is the final formula for the rolling ellipse center path length, and this formula will be give us $C(R,R)=2 \pi R$.

\section{Final Result}
The final formula for the length of the curve generated by the center of a rolling ellipse is given by:
\[
    \boxed{C(a,b)=\int_{0}^{2\pi}\sqrt{\big(a^2\sin^2 t+b^2\cos^2 t\big)
            +\left(\frac{dY}{dt}\right)^2}dt ;}
\]

\end{document}
